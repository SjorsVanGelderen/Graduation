% Graduation project documentation
% Copyright 2016, Sjors van Gelderen

% Document settings
\documentclass{article}
\author{Sjors van Gelderen}
\title{Exploring advanced programming concepts}
\date{\today{}}

% Packages
\usepackage{amsmath}
\usepackage[utf8]{inputenc}
\usepackage{graphicx}
\usepackage{listings}

% Content
\begin{document}

\maketitle{}
\tableofcontents{}

\section{Introduction}
\paragraph{}
In this document you will find a complete description of the concepts studied during the course of my graduation phase.

The primary goal of this project is to gain proficiency in the following fields:
\begin{itemize}
\item Data structures
\item Algorithms
\item Empirical analysis
\item Complexity analysis
\item Asynchronous programming
\item Advanced language features
\end{itemize}

In order to study different perspectives on the subject matter,
I have chosen to use multiple programming languages and paradigms.

For instance, although C\# and F\# may be used with the same paradigm,
C\#'s primary focus is on imperative, object-oriented programming;
whereas F\# is designed to better accommodate a structural, functional programming style.

\subsection{Python 3}
Due to its high level nature, the language offers a concise syntax that helps one rapidly express ideas.
The language does not require manual memory management or typing.
Python 3 snippets often look remarkably similar to pseudocode.
For all the comfort of the freedom it offers, the absence of a strict compiler often leads to unreliable programs.

\subsection{C\#}
With Microsoft joining the Linux Foundation, and making .NET Core available for every major operating system,
using .NET family of programming languages becomes ever more attractive.
Recent developments have made C\# more attractive to functional programming advocates.
C\# is a robust, high-level general purpose language.

\subsection{F\#}
One of the more recent additions to the .NET family of programming languages,
F\# offers a powerful functional programming syntax.
Its default immutability, static typing and strict compiler tend to result in extremely reliable programs.

\subsection{C}
C is known for being one of the most portable languages in existance.
It is a versatile, fast, low-level imperative programming language.
The main advantage C offers for this research is that it forces one to think about the way data is stored and manipulated in memory.
It gives a better overview of how the programmer's decisions affect the performance of the program.

\subsection{Rust}
Developed by Mozilla, Rust is a promising new programming language.
Its primary attraction factor lies in its strict syntax and accompanying compiler.
The concepts of borrowing and lifetime make asynchronous programming significantly more reliable.

\subsection{Chicken Scheme}
This language is part of the LISP-family of languages.
It has a very minimalistic syntax, revolving around the use of parentheses and prefix notation.
Like F\#, Chicken Scheme is a functional programming language.

\section{Analysis}

\subsection{Empirical analysis}
\paragraph{}
Empirical analysis refers to inferring a program's expected performance based on measurements taken while running a program with various configurations.
A typical scenario involves the measurement of performance in terms of running time by the use of a 'stopwatch'.

\subsection{Complexity analysis}
Complexity analysis is different from empirical analysis in that it uses reasoning
- rather than experiment - to determine a program's expected performance.

\section{Data structures}
\paragraph{}
The field of algorithms is closely tied to that of data structures. As the manner in which data is stored determines both the efficiency and the shape of a given algorithm, it is appropriate to discuss the applied data structures now.

\subsection{Linked list}
\paragraph{}
The linked list is a simple, but versatile data structure.
In essence, it consists of segments, each of which contains a value and a reference to the next segment.

The main advantages of linked lists are:
\begin{itemize}
\item Dynamically resizable
\item Fragmented storage, no contiguous memory required
\item Simple insertion, deletion and traversal operations
\end{itemize}

\subsection{Stack}
\subsection{Queue}
\subsection{Binary tree}
\subsection{Binary search tree}
\subsection{K-dimensional tree}
\subsection{Binary space partition tree}


\section{Algorithms}
\subsection{Dijkstra's algorithm}

\newpage{}

\end{document}
